% week number
\newcounter{schoolWeek}
\setcounter{schoolWeek}{1}

%%%%%%%%%%%%%%%%%%%%%%%%%%%%%%%%%%%%%%%%%%%%%%%%%

\documentclass[6pt]{scrbook}
\renewcommand*{\pagemark}{}% remove page number

% page margin
\usepackage[a6paper,margin=5mm]{geometry}

% german week day names
\usepackage[ngerman]{babel}
\usepackage[ngerman]{translator}

% calendar and bookmark
\usepackage{tikz}
\usetikzlibrary{calendar}
\usetikzlibrary{positioning}
\usetikzlibrary{calc}


\usepackage{colortbl} % colors
\usepackage{lmodern} % modern font
\usepackage{graphicx}
\usepackage{titling} % for creating titlepage

\usepackage{fontspec}
\setmainfont{Forte} % SELECT FONT

% Title and author
\title{Hausaufgabenheft}
\author{Max Mustermann}

% start and end date
\def\firstday{2024-08-26}
\def\lastday{2025-02-16}

% lessons (exactly 8) ~ for " "
\def\montag{Bio,Bio,E,E,M,M,F,F}
\def\dienstag{D,I,E,N,S,T,A,G}
\def\mittwoch{M,I,T,T,W,O,C,H}
\def\donnerstag{D,O,N,N,E,R,S,T}
\def\freitag{F,R,E,I,T,A,G,~}
\def\wochenende{N,O,T,I,Z,E,N,~} % import customizable data


%%%%%%%%%%%%%%%%%%%%%%%%%%%%%%%%%%%%%%%%%%%%%%%%%

% column height
\newlength{\rowHeight}
\setlength{\rowHeight}{4.5mm}

\newlength{\topSpacing}
\setlength{\topSpacing}{5.5mm}

% colors
\colorlet{lightLineColor}{gray!80}

%%%%%%%%%%%%%%%%%%%%%%%%%%%%%%%%%%%%%%%%%%%%%%%%%
%%%%%%%%%%%%%%%%% ACTUAL DESIGN %%%%%%%%%%%%%%%%%
%%%%%%%%%%%%%%%%%%%%%%%%%%%%%%%%%%%%%%%%%%%%%%%%%

\newcommand\emptypage{\clearpage\thispagestyle{empty}\hfill}

\newcommand\topbox[1]{
	% heading
	{%
			\noindent
			\begin{tikzpicture}
				\draw[very thick] (0,2.5mm) -- ++(0,2.5mm)-- ++(4em,0);
				\draw[very thick] (0,-2.5mm) -- ++(0,-2.5mm) -- ++(4em,0);
				\node[anchor=center]at (\textwidth/2,0) {\Huge\bfseries #1 };
				\draw[very thick] (\textwidth-4em,5mm) -- ++(4em,0) -- ++(0,-2.5mm);
				\draw[very thick] (\textwidth-4em,-5mm) -- ++(4em,0) -- ++(0,2.5mm);
			\end{tikzpicture}
		}\\[\topSpacing]%
}


\newcommand\tikzWeekDay[2]{%
	\begin{scope}[auto]
		% Top Line
		\draw (5em,\rowHeight) -- (\textwidth-3.5em,\rowHeight);
		\draw (\textwidth-3em,\rowHeight) -- ++(3em,0) -- ++(0,-0.5em);
		% Inner elements
		\foreach \cLesson [count=\i] in #2{
				\draw[lightLineColor] (0,\i*-\rowHeight+\rowHeight) -- ++(\textwidth-3.5em,0);
				\draw[lightLineColor] (\textwidth-3em,\i*-\rowHeight+\rowHeight) -- ++(3em,0) -- ++(0,0.5em);
				\node[anchor=south] at (\textwidth-1.5em,\i*-\rowHeight+\rowHeight) {\Large\bfseries\cLesson};
			}%
		% day name
		\draw[fill=white,draw=none] (0,0.5em) rectangle ++(5em,-3em);
		\node[anchor=center] at (\rowHeight,0) {\Huge\bfseries #1};
		% day name corners
		\draw[thick] (0,\rowHeight) -- ++(4em,0) -- ++(0,-1em);
		\draw[thick] (0,-\rowHeight) -- ++(4em,0) -- ++(0,1em);
	\end{scope}
}
%%%%%%%%%%%%%%%%%%%%%%%%%%%%%%%%%%%%%%%%%%%%%%%%%
%%%%%%%%%%%%%%%%%%%%%%%%%%%%%%%%%%%%%%%%%%%%%%%%%

\newcommand\weekday[2]{%
	\begin{tikzpicture}
		\tikzWeekDay{#1}{#2}
	\end{tikzpicture}
}

\newcommand\rightweekday[2]{%
	\begin{tikzpicture}
		\begin{scope}[xscale=-1]
			\tikzWeekDay{#1}{#2}
		\end{scope}
	\end{tikzpicture}
}


\newcommand*{\printSundayDate}[1]{%
	% determine julian day number
	\newcount\julianday%
	\pgfcalendardatetojulian{\%y0-\%m0-\%d0}{\julianday}%
	% compute week day (monday=0, ..., sunday=6)
	\newcount\julianweekday%
	\pgfcalendarjuliantoweekday{\julianday}{\julianweekday}%
	% compute value to add
	\newcount\toAdd%
	\pgfmathsetcount{\toAdd}{-\julianweekday + 6}%
	% compute date of sunday
	\pgfcalendardatetojulian{\%y0-\%m0-\%d0+\toAdd}{\julianday}%
	\pgfcalendarjuliantodate{\julianday}{\thisyear}{\thismonth}{\thisday}%
	% print date
	\thisday.\thismonth.\thisyear%
}

%%%%%%%%%%%%%%%%%%%%%%%%%%%%%%%%%%%%%%%%%%%%%%%%%

\begin{document}

{%\fontfamily{lmss}\selectfont
\begin{titlepage}
	\centering
	\vspace*{-2mm}
	{\scshape \begin{figure}[h]
			\centering
			\includegraphics[width=0.4\textwidth,height=0.2\textheight,keepaspectratio]{
				%%%%%%%%%%%%%%%%%%%%%%%%%%%%%%%%%%%%%%%%%%%%%%
				%%% Replace this with your logo image path %%%
				%%%%%%%%%%%%%%%%%%%%%%%%%%%%%%%%%%%%%%%%%%%%%%
				images/dummyLogo.png
				%%%%%%%%%%%%%%%%%%%%%%%%%%%%%%%%%%%%%%%%%%%%%%
				%%%%%%%%%%%%%%%%%%%%%%%%%%%%%%%%%%%%%%%%%%%%%%
			}
		\end{figure} \par}
	{\fontsize{16pt}{17.5}\fontfamily{lmss}\selectfont\bfseries \thetitle\par}
	\vfill
	{\scshape \begin{figure}[h]
			\centering
			\includegraphics[width=\textwidth,height=0.42\textheight, keepaspectratio]{
				%%%%%%%%%%%%%%%%%%%%%%%%%%%%%%%%%%%%%%%%%%%%%%%%%%%
				%%% Replace this with your timetable image path %%%
				%%%%%%%%%%%%%%%%%%%%%%%%%%%%%%%%%%%%%%%%%%%%%%%%%%%
				images/dummyImage.png
				%%%%%%%%%%%%%%%%%%%%%%%%%%%%%%%%%%%%%%%%%%%%%%%%%%%
				%%%%%%%%%%%%%%%%%%%%%%%%%%%%%%%%%%%%%%%%%%%%%%%%%%%
			}
		\end{figure} \par}
	\vfill
	{\fontsize{10pt}{11.5}\selectfont\bfseries von\par \theauthor\par}
	\vfill
	\vspace*{5mm}
\end{titlepage}}
% \fontfamily{lmss}\selectfont
\let\%=\pgfcalendarshorthand%
\pgfcalendar{cal}{\firstday}{\lastday}{%
	\ifdate{Monday,equals=\firstday}{%
		\newpage
		% heading
		\topbox{Woche~\arabic{schoolWeek} - \%d0.\%m0. bis \printSundayDate{\%y0-\%m0-\%d0}}
		\weekday{\%w.}{\montag}\\[\rowHeight]%
	}{}%
	\ifdate{Tuesday}{%
		\weekday{\%w.}{\dienstag}\\[\rowHeight]%
	}{}%
	\ifdate{Wednesday}{%
		\weekday{\%w.}{\mittwoch}
	}{}%
	\ifdate{Thursday}{%
		\newpage
		% heading
		\topbox{~}
		\stepcounter{schoolWeek}%
		\rightweekday{\%w.}{\donnerstag}\\[\rowHeight]%
	}{}%
	\ifdate{Friday}{%
		\rightweekday{\%w.}{\freitag}\\[\rowHeight]%
	}{}%
	\ifdate{Saturday}{%
		\rightweekday{}{\wochenende}
	}{}%
}%
\emptypage % Add as many empty pages as you like
\emptypage
\end{document}